
\cleartoverso

\PoemTitle{Chapultepec Blues}

\verseDedication{für Iván}

\begin{verse}

An jenem Morgen in Chapultepec\\
sah ich meine Gedanken wie Ameisen wandern;\\
sie trugen ihre Bedeutung, ihren Schatz weißer Larven,\\
in ein neues Versteck.

Zwischen Eis- und Zeitungsverkäufern\\
auf den asphaltierten Wegen am Schloss\\
erzähltest du mir von dem letzten Kadetten,\\
der sich 1847 hier,\\
in die mexikanische Fahne gewickelt,\\
stürzte vom Dach in den Tod,\\
um sich den \emph{yanquis} nicht zu ergeben.

Im türkischen Café am Eingang des Parkes\\
las ich im Satz meiner Mokka-Tasse:

Das Ameisenheer der Geschichte,\\
Infanterie des Geistes,\\
Fußvolk unserer Leidenschaft:\\
lass es marschieren in Reih und Glied,\\
oder in Chaosformation\\
auseinanderstieben,\\
lass es frei,

wärend wir,\\
gewickelt in die Fahne unserer Persönlichkeit,\\
wie ein Seufzer sinken,\\
zurück in das Versteck\\
wo wir die vergessenen Larven\\
der Ameisen sind.

\end{verse}

\clearpage

\PoemTitle{Chapultepec Blues}

\verseDedication{for Iván}

\begin{verse}

That morning in Chapultepec\\
I saw my thoughts wander like ants;\\
they carried their meaning, their treasure\\
of white larvae\\
into a new hiding place.

Between ice and newspaper vendors\\
on the tarmacked paths around the castle\\
you told me about the last cadet\\
who, in 1847,\\
wrapped in the Mexican flag\\
jumped from its roof to his death\\
to avoid surrender to the \emph{yanquis}.

In the Turkish café near the park entrance\\
I read in the dregs of my mocha cup:

The ant army of history,\\
infantry of the mind,\\
foot-soldiers of our passions:\\
let it march in rank and file,\\
or scatter in chaos-formation,\\
let it go free,

while we,\\
wrapped in the flag of our personality,\\
sink like a sigh\\
back into the hiding place,\\
where we are the larvae\\
the ants forgot.

\end{verse}

