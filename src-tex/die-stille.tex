
\cleartorecto

\PoemTitle{Die Stille in Mailand}

\hspace{2em}\emph{für Chandra}

\begin{verse}

Stille der Großstadt,\\
ungelenke Eroberin,\\
sanfter Clown, Alltagsclown,\\
der horcht und der hört.

Stille mit den weit geöffneten Augen,\\
staunende Stille,\\
die uns an ihren warmen\\
Händen führt --

an den Stadtrand, wo\\
die Straßen sich verlaufen,\\
wo sie enden, in Halden\\
und Abstellplätzen,

an einer stehengelassenen\\
Mauer, oder\\
in einem stummen Feld.\\
Mosaik der Stille,

zusammengesetzt\\
aus weggeworfenen\\
Requisiten des Alltags,\\
aus Gesten und Blicken

die vorübergingen\\
im Zentrum der Stadt;\\
wo was wir tun,\\
was wir nehmen

oder liegenlassen,\\
zum stillen Kunstwerk deiner Sprache wird.\\
Plötzlich bleibst du stehen\\
und siehst mich an:

stille Rose am Mantelkragen,\\
Schneefall und aufgelesener Schirm.\\
Die Angst frisst dir aus den Händen,\\
deine Stimme ist firm.

\end{verse}

\cleartoverso

\PoemTitle{Silence in Milan}

\hspace{2em}\emph{for Chandra}

\begin{verse}

Silence of the big city,\\
clumsy conqueress,\\
gentle clown, everyday-clown,\\
that listens and hears.

Silence with wide open eyes,\\
marvelling silence\\
that leads us\\
with warm hands --

to the outskirts where\\
the streets lose themselves,\\
where they end in skips\\
and parking bays,

at left-behind walls\\
or in mute fields.\\
Mosaic of silence,\\
put together

from thrown-away requisites\\
of daily life,\\
from gestures and glances of those\\
who passed us

in the city-centre,\\
where everything we do,\\
everything we pick up\\
or leave behind

becomes the silent art\\
of your language.\\
Suddenly you stop\\
and look at me:

silent rose on the lapel of your coat,\\
snowfall and umbrella somewhere found.\\
Fear eats out of your hand,\\
your voice holding its ground.

\end{verse}
